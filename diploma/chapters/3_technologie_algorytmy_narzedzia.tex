\chapter{Technologie, algorytmy i narzędzia}
\label{chap:algs}

\section{Użyte narzędzia}

Silnik została napisany jako biblioteka w języku C w standardzie C11. Budowanie biblioteki ze źródeł wymaga generacji
dodatkowego kodu przy pomocy skryptów w języku Python w wersji 3.9.7.

Silnik został opracowany na maszynie z systemem Linux i wspiera systemy Linux i Windows. Projekt został w całości
napisany przy użyciu środowiska programistycznego CLion w wersji 2021.2.3.

Podczas pracy stosowano rozproszony system kontroli wersji git. Repozytorium jest utrzymywane na serwisie GitHub.

Pliki \textit{.clang-tidy} i \textit{.clang-format} znajdujące się w strukturze plików projektu pozwalają na automatyczne formatowanie
kodu źródłowego zgodnie ze uprzednio zdefiniowanym standardem kodowania.

Proces budowania projektu jest zautomatyzowany przy użyciu narzędzia CMake, które w przypadku języków C i C++ jest
praktycznie standardem podczas rozwoju wieloplatformowych projektów.

\subsection{Proces budowania}

// TODO


\subsection{Biblioteki zewnętrzne}

// TODO