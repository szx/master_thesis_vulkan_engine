\chapter*{Streszczenie}
\addcontentsline{toc}{chapter}{Streszczenie}
Cieniowanie odroczone jest techniką grafiki 3D czasu rzeczywistego popularną wśród twórców gier komputerowych pozwalającą na obsługę wielu światłen na scenie bez znaczącego spadku wydajności.

W niniejszej pracy zaprojektowano i zaimplementowano silnik renderujący używając języka C i biblioteki graficznej Vulkan.
Opisano elementy silnika renderującego oraz nisko i wysokopoziomowe techniki graficzne używane w nowoczesnych grach 3D z naciskiem na renderowanie odroczone. Opisano architekturę silnika i szczegóły implementacji. Wyrenderowano przykładową scenę i zbadano wydajność użytych technik graficznych.
\\
\\
Słowa kluczowe: silnik renderujący, renderowanie odroczone, renderowanie bez dowiązań, renderowanie pośrednie, Vulkan, glTF.
\\
Dziedzina nauki i techniki według OECD: 1.2 Nauki o komputerach i informatyka.

\chapter*{Abstract}
\addcontentsline{toc}{chapter}{Abstract}  
// TODO