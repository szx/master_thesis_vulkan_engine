\chapter*{Streszczenie}
\addcontentsline{toc}{chapter}{Streszczenie}
Cieniowanie odroczone jest techniką grafiki 3D czasu rzeczywistego popularną wśród twórców gier komputerowych pozwalającą na obsługę wielu świateł na scenie bez znaczącego spadku wydajności.

W niniejszej pracy zaprojektowano i zaimplementowano silnik renderujący używając języka C i biblioteki graficznej Vulkan.
Opisano elementy silnika renderującego oraz nisko i wysokopoziomowe techniki graficzne używane w nowoczesnych grach 3D z naciskiem na renderowanie odroczone. Opisano architekturę silnika i szczegóły implementacji. Wyrenderowano przykładową scenę i zbadano wydajność użytych technik graficznych.
\\
\\
Słowa kluczowe: silnik renderujący, cieniowanie odroczone, renderowanie bez dowiązań, renderowanie pośrednie, Vulkan, glTF.
\\
Dziedzina nauki i techniki według OECD: 1.2 Nauki o komputerach i informatyka.

\chapter*{Abstract}
\addcontentsline{toc}{chapter}{Abstract}  

Deferred shading is a real-time 3D graphics technique popular among computer game developers. It allows multiple lights on a stage without a significant drop in performance.

In this thesis, a rendering engine was designed and implemented using the C language and the Vulkan graphics library.
The elements of the rendering engine as well as low and high-level techniques used in modern 3D games with an emphasis on deferred rendering are described.
The engine architecture and implementation are described.
A sample scene was rendered and the performance was measured.
\\
\\
Keywords: rendering engine, deferred shading, bindless rendering, indirect rendering, Vulkan, glTF.
\\
Field of science and technology classification: 1.2 Computer and information sciences.