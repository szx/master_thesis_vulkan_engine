\chapter{Wstęp}
\label{chap:introduction}

Podręcznik \cite{HughesDamEtAl13} definiuje grafikę komputerową jako dziedzinę interdyscyplinarna zajmującą się komunikacją wizualną za pomocą wyświetlacza komputera i jego urządzeń wejścia-wyjścia.

Rozwój teoretyczny grafiki komputerowej może być śledzony na dorocznej konferencji SIGGRAPH \cite{SIGGRAPH}, podczas której prezentacje i dyskusje akademickie są przeplecione z targami branżowymi.
Rozwój grafiki komputerowy jest w znacznej mierze napędzany wymaganiami stawianymi przez przemysł rozrywkowy. Przykładem jest dobrze znana seria kursów przeznaczona dla twórców gier komputerowych obejmująca najnowsze prace i postępy w technikach renderowania czasu rzeczywistego używanych w silnikach graficznych rozwijanych przez producentów gier komputerowych \cite{SIGGRAPH_ADVANCES}.

Renderowanie to proces konwersji pewnych prymitywów na obraz przeznaczony do wyświetlania na ekranie.
Wyświetlany obraz jest nazywany klatką (ang. frame).

Renderowanie czasu rzeczywistego nakłada ograniczenie czasowe dotyczące liczby klatek na sekundę (ang. frames per second, FPS), która musi być na tyle wysoka, by dawać iluzję ciągłości ruchu.
Przyjmuje się, że ograniczenie to jest spełniane poprzez zapewnienie wyświetlanie minimum 30 klatek na sekundę (renderowanie trwa krócej niż $\frac{1}{30}$ sekundy).
Eliminuje to kosztowne obliczeniowo techniki renderowania dające fotorealisteczne rezultaty takie jak śledzenie promieni (ang. ray tracking) i wymaga od programistów zastosowania technik aproksymacji mniej lub bardziej luźno opertych na prawach fizyki oraz użycia bibliotek graficznych wspierających akcelerację sprzętową takich jak Vulkan lub OpenGL.

Silnik renderujący, zwany też silnikiem graficznym to element aplikacji odpowiadający za renderowanie czasu rzeczywistego. Zapewnia on wysokopoziomową warstwę abstrakcji pozwalającą użytkownikowi na operowanie używając takich konceptów jak sceny, obiekty, materiały lub światła oraz ukrywają niskopoziomowe detale użytych bibliotek i technik graficznych.

Zaprojektowanie i zaimplementowanie silnika graficznego jest złożonym procesem wymagającym znajomości szerokiego wachlarza technik graficznych z całego możliwego spektrum poziomów abstrakcji, dlatego też coraz więcej twórców gier komputerowych decyduje się na licencjonowanie i użycie gotowego silnika graficznego zamiast powolnej i mozolnej pracy nad własnymi rozwiązaniami.

Jeśli jednak celem inżyniera jest poszerzenie osobistego zrozumienia grafiki komputerwej, to warto podjąć próbę stworzenia własnego silnika graficznego.


\section{Cel pracy}

Celem pracy jest zaprojektowanie i zaimplementowanie silnika graficznego, którego potok graficzny używa techniki cieniowania odroczonego.

Silnik został napisany jako biblioteka programistyczna języka C używającej skryptów Python do automatycznej generacji kodu podczas procesu budowania.
Renderowanie grafiki 3D jest obsługiwane przez Vulkan API.
Zasoby używane podczas renderowania są wczytywane z bazy zasobów, która jest wyjściem potoku zasobów działającego podczas procesu budowania.
Działanie biblioteki jest sterowane plikiem konfiguracyjnym dostarczonym przez użytkownika i jest demonstrowane przy użyciu pliku wykonywalnego renderujacego przykładową scenę używając cieniowania odroczonego.

Celem autora było zapoznanie się z teorią stojącą za elementami składającymi się na silnik graficzny i praktyczne zademonstrowanie zdobytej wiedzy.


\section{Zakres pracy}

Niniejsza praca ma charakter przeglądowy. Nowoczesne silniki graficzne składają się z wielu elementów, z których każdy może być niezależnie rozwijany do dowolnie wysokiego poziomu skomplikowania, dlatego trudno je wszystkie dokładnie i wyczerpująco opisać w ramach jednej pracy.

Zakres pracy obejmuje:
\begin{itemize}
	\item {opis algorytmów i technik graficznych używanych w nowoczesnych silnikach graficznych, ze szczególnym naciskiem na Vulkan API i renderowanie odroczone},
	\item{omówienie architektury i implementacji projektu},
	\item{demonstrację użycia silnika graficznego do wyrenderowania przykładowej sceny},
	\item{analizę wydajności silnika graficznego}.
\end{itemize}

Stworzony silnik nie może konkurować z silnikami graficznymi profesjonalnie rozwijanymi przez duże drużyny z myślą o zastosowaniu w grach komputerowych (takimi jak otwarty Godot \cite{godotengine} czy komercyjny Unreal Engine \cite{unrealengine}).
Jest on jednak przystosowany do względnie łatwej, szybkiej i elastycznej modyfikacji potoku graficznego oraz wspiera mechanizmy zgłaszania informacji debugowania oferowane przez Vulkan API, co pozwala na szybki cykl prototypowania i debugowania podczas zapoznawania się z technikami graficznymi.

\section{Struktura pracy}


Praca została podzielona na pięć rozdziałów, z których każdy jest rozwinięciem rozdziału poprzedniego.

Pierwszy rozdział pracy definiuje cel, zakres i strukturę pracy.

Drugi rozdział zawiera wprowadzenie do wybranych części dziedziny grafiki komputerowej użytych podczas późniejszej implementacji silnika renderującego.

W trzecim rozdziale opisano architekturę silnika i szczegóły implementacji poszczególnych jego modułów.

W czwartym rozdziale wyrenderowano przykładową scenę używając potoku graficznego używającego cieniowania odroczonego  i zbadano wydajność silnika.

Ostatni rozdział zawiera podsumowanie oraz opis przewidywanych kierunków przyszłego rozwoju silnika.




