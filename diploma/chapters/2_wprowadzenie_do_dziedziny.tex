\chapter{Wprowadzenie do dziedziny}
\label{chap:field}

W tej sekcji przybliżono serię pojęć oraz technik związanych z grafiką komputerową, których zrozumienie jest wymagane przed rozpoczęciem implementacji silnika graficznego.

\section{Podstawowe pojęcia}
// TODO

\section{Potok graficzny}
// TODO

\section{Potok zasobów}
// TODO

\section{Vulkan}

// TODO

\subsection{Synchronizacja}

// TODO

\subsubsection{Bariery potoku}

Bariera potoku to prymityw synchronizacji definiowany poleceniem \textit{vkCmdPipelineBarrier()} pozwalający na zdefiniowanie zależności
wykonania oraz zależności pamięci pomiędzy poleceniami przed i po barierze.

Zależność wykonania to gwarancja, że praca pewnych \textit{źródłowych etapów potoku} (określonych używając
VkPipelineStageFlags) dla wcześniejszego zestawu poleceń została zakończona przed rozpoczęciem wykonywania pewnych
\textit{docelowych etapów potoku} dla późniejszego etapu poleceń. 

Przykładowo zależność wykonania pomiędzy etapami
COLOR\_ATTACHMENT\_OUTPUT i FRAGMENT\_SHADER gwarantuje, że zapis do dołączeń kolorów został skończony przed rozpoczęciem
wykonywania shadera fragmentów.

Zależność pamięci to zależność wykonania z dodatkową gwarancją, że rezultat zapisów wyspecyfikowanych przez pewien \textit{źródłowy
zakres dostępów} (określony używając VkAccessFlags) wygenerowanych przez wcześniejszy zestaw poleceń jest udostępiony
późniejszemu zestawowi poleceń dla pewnego \textit{docelowego zakresu dostępów}.

Przykładowo zaleźność pamięci pomiędzy etapami
COLOR\_ATTACHMENT\_OUTPUT i FRAGMENT\_SHADER z zakresami dostępów COLOR\_ATTACHMENT\_WRITE i SHADER\_READ gwarantuje, że zapis
do dołączeń kolorów zostanie skończony i będzie mógł być odczytany przez shader fragmentów.

Istnieją trzy rodzaje barier w zależności od rodzaju pamięci zarządzanego przez zależności pamięci:

\begin{itemize}
	\item {bariery pamięci obrazów}: dla zakresu obrazu, dodatkowo pozwala na tranzycje układu obrazu,
	\item {bariery pamięci buforów}: dla zakresu bufora,
	\item {globalne bariery pamięci}: dla wszystkich istniejących obiektów,
\end{itemize}

// TODO


\subsubsection{Semafory}

Semafory to obiekty VkSemaphore pozwalające na synchronizację wykonywania buforów poleceń w tej samej lub pomiędzy kolejkami. GPU
może sygnalizować semafor po zakończeniu wykonywania poleceń oraz może czekać na sygnalizację semafora przed
rozpoczęciem wykonywania następnego bufora poleceń.

Przykładowo semafory są używane do synchronizacji pomiędzy kolejką
graficzna i kolejką prezentacji w celu zagwarantowania, że prezentowalny obraz łańcucha wymiany jest używany tylko przez jedną kolejkę.

\subsubsection{Ogrodzenia}

Ogrodzenia to obiekty VkFence pozwalające na synchonizację poleceń wykonywanych w kolejce na GPU z CPU.
Ogrodzenie może być sygnalizowane przez GPU po zakończeniu wykonywania funkcji używających GPU, CPU może zresetować ogrodzenie
funkcją \textit{vkResetFences()} lub czekać na jego sygnalizację funkcją \textit{vkWaitForFences()} chwilowo blokując wykonywanie programu.

Przykładowo ogrodzenia są używane do zagwarantowania, że program nie używa funkcji \textit{vkQueueSubmit()} do wysyłania buforów poleceń szybciej, niż GPU je wykonuje.

\subsection{Deskryptory}

// TODO


\section{Mapowanie tekstur}

// TODO

\section{Oświetlenie}

// TODO

\section{Graf sceny}

// TODO

\section{Graf renderowania}

// TODO