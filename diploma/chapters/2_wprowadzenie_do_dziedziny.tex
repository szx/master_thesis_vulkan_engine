\chapter{Wprowadzenie do dziedziny}
\label{chap:field}

W tej sekcji przybliżono serię pojęć oraz technik związanych z grafiką komputerową, których zrozumienie jest wymagane przed rozpoczęciem implementacji silnika graficznego.

\section{Podstawowe pojęcia}
// TODO

\section{Potok zasobów}
// TODO

\subsection{glTF}

// TODO

\section{Vulkan}

// TODO HISTORIA, core vs ext, promowanie

\subsection{Podstawy API}

// TODO użycie api, notacje, łańcuch pNext

\subsection{Inicjalizacja}

// TODO surface, urządzenie fizyczne i logiczne

\subsection{Bufory poleceń}

// TODO surface, urządzenie fizyczne i logiczne

\subsection{Zasoby}
// TODO pamięć
// TODO bufory uniform i bufory magazynowe, obrazy, próbniki, obrazy magazynowe

\subsection{Łańcuch wymiany}

// TODO surface

\subsection{Synchronizacja}

// TODO

\subsubsection{Bariery potoku}

Bariera potoku to prymityw synchronizacji definiowany poleceniem \textit{vkCmdPipelineBarrier()} pozwalający na zdefiniowanie zależności
wykonania oraz zależności pamięci pomiędzy poleceniami przed i po barierze.

Zależność wykonania to gwarancja, że praca pewnych \textit{źródłowych etapów potoku} (określonych używając
VkPipelineStageFlags) dla wcześniejszego zestawu poleceń została zakończona przed rozpoczęciem wykonywania pewnych
\textit{docelowych etapów potoku} dla późniejszego etapu poleceń. 

Przykładowo zależność wykonania pomiędzy etapami
\textit{COLOR\_ATTACHMENT\_OUTPUT} i \textit{FRAGMENT\_SHADER} gwarantuje, że zapis do dołączeń kolorów został skończony przed rozpoczęciem
wykonywania shadera fragmentów.

Zależność pamięci to zależność wykonania z dodatkową gwarancją, że rezultat zapisów wyspecyfikowanych przez pewien \textit{źródłowy
zakres dostępów} (określony używając VkAccessFlags) wygenerowanych przez wcześniejszy zestaw poleceń jest udostępiony
późniejszemu zestawowi poleceń dla pewnego \textit{docelowego zakresu dostępów}.

Przykładowo zaleźność pamięci pomiędzy etapami
\textit{COLOR\_ATTACHMENT\_OUTPUT} i \textit{FRAGMENT\_SHADER} z zakresami dostępów \textit{COLOR\_ATTACHMENT\_WRITE} i \textit{SHADER\_READ} gwarantuje, że zapis
do dołączeń kolorów zostanie skończony i będzie mógł być odczytany przez shader fragmentów.

Istnieją trzy rodzaje barier w zależności od rodzaju pamięci zarządzanego przez zależności pamięci:

\begin{itemize}
	\item {bariery pamięci obrazów}: dla zakresu obrazu, dodatkowo pozwala na tranzycje układu obrazu,
	\item {bariery pamięci buforów}: dla zakresu bufora,
	\item {globalne bariery pamięci}: dla wszystkich istniejących obiektów,
\end{itemize}

// TODO użycie
// TODO listingi?

\subsubsection{Semafory}

Semafory to obiekty VkSemaphore pozwalające na synchronizację wykonywania buforów poleceń w tej samej lub pomiędzy kolejkami. GPU
może sygnalizować semafor po zakończeniu wykonywania poleceń oraz może czekać na sygnalizację semafora przed
rozpoczęciem wykonywania następnego bufora poleceń.

Przykładowo semafory są używane do synchronizacji pomiędzy kolejką
graficzna i kolejką prezentacji w celu zagwarantowania, że prezentowalny obraz łańcucha wymiany jest używany tylko przez jedną kolejkę.

\subsubsection{Ogrodzenia}

Ogrodzenia to obiekty VkFence pozwalające na synchonizację poleceń wykonywanych w kolejce na GPU z CPU.
Ogrodzenie może być sygnalizowane przez GPU po zakończeniu wykonywania funkcji używających GPU, CPU może zresetować ogrodzenie
funkcją \textit{vkResetFences()} lub czekać na jego sygnalizację funkcją \textit{vkWaitForFences()} chwilowo blokując wykonywanie programu.

Przykładowo ogrodzenia są używane do zagwarantowania, że program nie używa funkcji \textit{vkQueueSubmit()} do wysyłania buforów poleceń szybciej, niż GPU je wykonuje.

\subsection{Deskryptory i stałe push}

Vulkan nie pozwala na bezpośredni dostęp do zasobów z poziomu shadera i wymaga użycia deskryptorów.

Deskryptor to blok pamięci z opisem pojedyńczego zasobu używanego przez GPU. Dokładna wewnętrzna struktura deskryptora jest w formacie specyficznym dla GPU, ale może być intuicyjnie rozumiana jako struktura zawierająca wskaźnik to adresu pamięci GPU z danymi zasobu oraz dodatkowe metadane opisujące rodzaj zasobu oraz w jaki sposób zasób będzie używany przez shader.

\subsubsection{Tworzenie deskryptorów}

Vulkan nie pozwala na tworzenie i używanie pojedyńczych deskryptorów i wymaga grupowania ich w tablice poprzez zbiory deskryptorów  (obiekt \textit{VkDescriptorSet}).

Stworzenie zbiorów deskryptorów wymaga wcześniejszego stworzenia dwóch obiektów: puli deskryptorów (\textit{VkDescriptorPool})
oraz układu zbioru deskryptorów (\textit{VkDescriptorSetLayout}).

Pula deskryptorów to źródło, z którego alokowane są deskryptory w postaci zbiorów deskryptorów. Podczas tworzenia należy zadeklarować:
\begin{itemize}
	\item maksymalną liczbę zaalokowanych zbiorów deskryptorów,
	\item maksymalną liczbę rodzajów deskryptorów.
\end{itemize}

Układ zbioru deskryptorów reprezentuje wewnętrzną strukturę zbioru deskryptorów - programista języka C może o nim myśleć jako o deklaracji struktury używanej później do definiowania zmiennych (zbiorów deskryptorów).
Układ składa się z listy dowiązań deskryptorów (\textit{VkDescriptorSetLayoutBinding}).

Jedno dowiązanie deskryptora reprezentuje fragment zbioru deskryptorów zajmowany przez deskryptory tego samego typu.
Każde dowiązanie deskryptora jest opisane poprzez:
\begin{itemize}
	\item {\textit{numer dowiązania}}: używany do odnoszenia się w shaderze do dowiązania i uzyskania dostępu do
	zasobu,
	\item \textit{typ deskryptora},
	\item \textit{liczba deskryptorów},
	\item {\textit{zbiór etapów cieniowania}}: określa które shadery w potoku graficznym mają dostep do zasobów.
\end{itemize}
Typ deskryptora zależy od rodzaju opisywanego zasobu, przykładowo:
\begin{itemize}
	\item {\textit{UNIFORM\_BUFFER}}: bufor uniform,
	\item {\textit{UNIFORM\_BUFFER\_DYNAMIC}}: dynamiczny bufor uniform, dodatkowy dynamiczny offset jest specyfikowany podczas dowiązywania zbioru deskryptorów,
	\item {\textit{STORAGE\_BUFFER}}: bufor magazynowy,
	\item {\textit{STORAGE\_BUFFER\_DYNAMIC}}: dynamiczny bufor magazynowy,
	\item {\textit{SAMPLER}}: próbnik,
	\item {\textit{SAMPLED\_IMAGE}}: widok próbkowalnego obrazu,
	\item {\textit{STORAGE\_IMAGE}}: widok obrazu magazynowego,
	\item {\textit{COMBINED\_IMAGE\_SAMPLER}}: próbkowany obraz, pojedyńczy deskryptor jest skojarzony zarówno z próbnikiej, jaki i z widokiem obrazu,
	\item {\textit{UNIFORM\_TEXEL\_BUFFER}}: widok bufora uniform,
	\item {\textit{STORAGE\_TEXEL\_BUFFER}}: widok bufora magazynowego.
\end{itemize}

\subsubsection{Aktualizacja deskryptorów}

Po stworzeniu zbioru deskryptorów zawartość jego deskryptorów jest niezdefiniowna i musi być zaktualizowana funkcją
\textit{vkUpdateDescriptorSets()}. Jej wejściem jest \textit{tablica struktur VkWriteDescriptorSet}, której każdy pojedyńczy element opisuje
który wycinek tablicy wybranego dowiązania w zbiorze deskryptorów powinien być zaktualizowany informacjami o zasobach.

Aktualizacja zbioru deskryptorów odbywa się na CPU natychmiastowo po wywołaniu \textit{vkUpdateDescriptorSets()} i jest możliwa
tylko zanim zbiór deskryptorów zostanie użyty przez jakiekolwiek polecenie w nagrywanym bądź wykonywanym buforze poleceń.
Jednym z wyjątków jest aktualizacja zbiorów deskryptorów zaalokowanych z puli deskryptorów wspierającej funkcjonalność uaktualnienia deskryptorów po dowiązaniu.

\subsubsection{Stałe push}

Stałe push to sposób przekazywania danych do shaderów będący szybszą i łatwiejszą 
alternatywą dla deskryptorów. Nie wymagają one tworzenia i aktualizacji zasobów opartych na pamięci GPU - pamięć CPU stałej push jest bezpośrednio kopiowana i przechowywana w nagrywanym buforze poleceń komendą \textit{vkCmdPushConstants()}.

Niestety ta metoda ma poważne ograniczenie - minimalny rozmiar pamięci udostępniany shaderowi gwarantowany przez specyfikację Vulkan to tylko 128 bajtów, co odpowiada dwóm macierzom 4x4. Z tego powodu stałe push powinny być używane do przekazywania danych, które zmieniają się na tyle często, że narzut wydajnościowy synchronizacji modyfikowanych buforów uniform. Przykładem mogą być macierze transformacji albo indeksy tekstur używane przez polecenia rysowania.


\subsubsection{Zadeklarowanie użycia deskryptorów w układzie potoku}

Układ potoku (VkPipelineLayout) zawiera informacje o sposobie organizacji wszystkich zbiorów deskryptorów i stałych push, które mogą być używane w potoku (VkPipeline). Jest on używany do dowiązywania zbiorów deskryptorów i nagrywania stałych push.

Podczas tworzenia należy zadeklarować:
\begin{itemize}
	\item listę układów zbiorów deskryptorów,
	\item listę zakresów stałych push (VkPushConstantRange).
\end{itemize}

Zakres stałej push składa się z:
\begin{itemize}
	\item zbioru etapów cieniowania mających dostęp do stałej push,
	\item offset i rozmiar pamięci, który moze być używany przez powyższe etapy cieniowania.
\end{itemize}

Liczba poszczególnych typów deskryptorów uzwględnionych w potoku renderowania jest ograniczona limitami urządzenia fizycznego.
Limit \textit{maxPerStageResources} to maksymalna liczba zasobów, które mogą być dostępne dla pojedyńczego etapu cieniowania.
Rodzina limitów \textit{maxDescriptorSet*}, gdzie \textit{*} to typ deskryptora, kontroluje maksymalną liczbę deskryptorów danego typu w układzie potoku.

\subsubsection{Dowiązanie deskryptorów do bufora poleceń}

Przed użyciem zasobów opisanych zbiorem deskryptorów przez polecenia rysowania wymagane jest dowiązania ich do bufora
poleceń przy użyciu komeny \textit{vkCmdBindDescriptorSets()}. Jednym z jej wejść jest \textit{numer zbioru}, który wraz z numerami
dowiązań służy do identyfikacji zasobu w shaderach.


\subsubsection{Diagram użycia deskryptorów}

Relacje pomiędzy obiektami, funkcjami i komendami używającymi deskryptorów i stałych push zostały przedstawione na poniższym diagramie:
\begin{figure}[H]
	\centering
	\begin{tikzpicture}[node distance=1cm]
		\tikzstyle{object} = [rectangle, minimum width=3cm, minimum height=1cm,text centered, draw=black]
		\tikzstyle{function} = [rectangle, rounded corners,text centered, draw=black]
		\tikzstyle{arrow} = [thick,->,>=stealth]
		\tikzstyle{relation} = [densely dotted]

		\node (VkDescriptorSetLayout) [object] {VkDescriptorSetLayout};
		\node (VkDescriptorPool) [object, right = of VkDescriptorSetLayout] {VkDescriptorPool};
		\node (VkDescriptorSet) [object, below = of VkDescriptorPool] {VkDescriptorSet};
		\draw [arrow] (VkDescriptorSetLayout) -- (VkDescriptorSet);
		\draw [arrow] (VkDescriptorPool) -- (VkDescriptorSet);
		
		\node (VkPipelineLayout) [object, below = of VkDescriptorSetLayout] {VkPipelineLayout};
		\node (VkPushConstantRange) [object, left = of VkDescriptorSetLayout] {VkPushConstantRange};
		\draw [arrow] (VkDescriptorSetLayout) -- (VkPipelineLayout);
		\draw [arrow] (VkPushConstantRange) -- (VkPipelineLayout);
		
		\node (vkUpdateDescriptorSets) [function, right = of VkDescriptorSet] {vkUpdateDescriptorSets};
		\draw [arrow] (VkDescriptorSet) -- (vkUpdateDescriptorSets);
		\draw [arrow] (vkUpdateDescriptorSets) -- (VkDescriptorSet);
		\node (resources) [object, rectangle split, rectangle split parts=2, above = of vkUpdateDescriptorSets] {VkBuffer \nodepart{two} VkImage};
		\draw [arrow] (resources) -- (vkUpdateDescriptorSets);
		\draw [relation] (resources) edge[out=135, in=45, looseness=0.4] (VkDescriptorSetLayout);
		
		\node (vkCmdBindDescriptorSets) [function, below = of VkDescriptorSet] {vkCmdBindDescriptorSets};
		\draw [arrow] (VkPipelineLayout) -- (vkCmdBindDescriptorSets);
		\draw [arrow] (VkDescriptorSet) -- (vkCmdBindDescriptorSets);
		
		\node (vkCmdPushConstants) [function, below = of VkPipelineLayout] {vkCmdPushConstants};
		\draw [arrow] (VkPipelineLayout) -- (vkCmdPushConstants);
		\node (pushmemory) [object, text width=2cm, left = of vkCmdPushConstants] {pamięć CPU stałej push};
		\draw [arrow] (pushmemory) -- (vkCmdPushConstants);
		\draw [relation] (pushmemory) -- (VkPushConstantRange);
		
		\node (VkCommandBuffer) [object, below = of vkCmdBindDescriptorSets] {VkCommandBuffer};
		\draw [arrow] (vkCmdPushConstants) -- (VkCommandBuffer);
		\draw [arrow] (vkCmdBindDescriptorSets) -- (VkCommandBuffer);
		
		 \path ([xshift=34mm,yshift=-2mm]current bounding box.south west)
		node[matrix,anchor=south east,cells={nodes={font=\sffamily,anchor=west}},
		draw,thick,inner sep=1ex]{
			\draw[arrow](0,0) -- ++ (0.6,0); & \node{Użycie};\\
			\draw[relation](0,0) -- ++ (0.6,0); & \node{Kompatybilność};\\
		};

	\end{tikzpicture}
	\caption{Relacje pomiędzy obiektami Vulkan używanymi do zarządzania deskryptorami}
	\label{descriptor_relations}
\end{figure}

\subsubsection{Dostęp do zasobów w shaderach}

Po dowiązaniu zbiorów deskryptorów i stałych push do bufora poleceń dostęp do zasobów z poziomu kodzu GLSL shadera odbywa się poprzez zmienną posiadającą odpowiednie kwalifikatory układu.

Przykładowo kwalifikator układu dla pojedyńczego deskryptora typu UNIFORM\_BUFFER z dowiązania o numerze $x$ ze zbioru o numerze $y$ ma następującą formę:
\lstset{language=GLSL}
\begin{lstlisting}
struct bufferStruct {
	vec3 field1;
	mat4 field2;
	...
};
layout(scalar, set = y, binding = x) uniform bufferBlock {
	bufferStruct buffer;
};
\end{lstlisting}

Analogicznie kwalifikator układu dla tablicy deskryptorów typu COMBINED\_IMAGE\_SAMPLER o rozmiarze $r$ z dowiązania o numerze $x$ ze zbioru o numerze $y$ próbkowanych obrazów 2D ma następującą formę:
\lstset{language=GLSL}
\begin{lstlisting}
layout(set = y, binding = x) uniform sampler2D texture[r];
\end{lstlisting}

\section{Rozszerzenie VK\_EXT\_descriptor\_indexing}

Rozszerzenie \textit{VK\_EXT\_descriptor\_indexing} wprowadziło szereg dodatkowych funkcjonalności pozwalających na tworzenie dużych zbiorów
deskryptorów zawierających wszystkie zasoby używane przez program.
Celem tego jest umożliwienie technik renderowania bez dowiązań. Z powodu swojej użyteczności rozszerzenie to zostało promowane w Vulkan 1.2.
W kolejnych sekcjach opisano nowe funkcjonalości.

\subsection{Niejednolite dynamiczne indeksowanie deskryptorów}

Deskryptory są traktowane przez shadery jako tablice, do których dostęp odbywa się używając indeksu.

Statyczne indeksowanie pozwala na dostęp do zasobu przy użyciu indeksu będącego stałą czasu kompilacji. Jest to najstarszy i zawsze wspierany sposób indeksowania.

Dynamiczne indeksowanie pozwala na dostęp do zasobu przy użyciu wartości czasu wykonywania.

Jednolite dynamiczne indeksowanie wymaga, żeby indeks był taki sam we wszystkich wywołań shadera spowodowanych przez pojedyńcze polecenie rysowania - użycie różnych indeksów jest błędem i może skutkować korupcjami.
Przykładem tego rodzaju indeksowania jest użycie indeksu tekstury w stałej push lub buforze uniform.
Wymaga ona wsparcia następujących funkcjonalności urządzenia Vulkan 1.0:
\begin{itemize}
	\item {\textit{shaderUniformBufferArrayDynamicIndexing}}: tablice buforów uniform,
	\item {\textit{shaderSampledImageArrayDynamicIndexing}}: tablice próbkowalnych obrazów,
	\item {\textit{shaderStorageBufferArrayDynamicIndexing}}: tablice buforów magazynowych,
	\item {\textit{shaderStorageImageArrayDynamicIndexing}}: tablice obrazów magazynowych.
\end{itemize}

Niejednolite dynamiczne indeksowanie pozwala na swobodny dostęp do zasobów znajdujących się w pamięci GPU przy użyciu dowolnych indeksów.
Przykładem może być indeksowanie tablicy tekstur używając indeksu instancji polecenia rysowania albo próbki tekstury pozaekranowej.
Wymaga ona wsparcia analogicznych funkcjonalności urządzenia Vulkan 1.2:
\begin{itemize}
	\item {\textit{shaderUniformBufferArrayNonUniformIndexing}},
	\item {\textit{shaderSampledImageArrayNonUniformIndexing}},
	\item {\textit{shaderStorageBufferArrayNonUniformIndexing}},
	\item {\textit{shaderStorageImageArrayNonUniformIndexing}}.
\end{itemize}

W czasie pisania pracy powyższe funkcjonalności są wspierane przez więcej niż $90\%$ urządzeń \cite{GPUINFO}.

Niejednolite dynamiczne indeksowanie wymaga zadeklarowania rozszerzenia SPIR-V \textit{SPV\_EXT\_descriptor\_indexing}.
Może być to uczynione z poziomu kodu GLSL przez dodanie następującej dyrektywy preprocesora:
\lstset{language=GLSL}
\begin{lstlisting}
#extension GL_EXT_nonuniform_qualifier : require
\end{lstlisting}
Dodatkowo każde użycie niejednolitego indeksu powinno być oznaczone funkcją \textit{nonuniformEXT}:
\lstset{language=GLSL}
\begin{lstlisting}
vec4 color = texture(textures2D[nonuniformEXT(index)], uv);
\end{lstlisting}

Wymóg jednolitości podczas indeksowania deskryptorów był spowodowany ograniczeniami poprzednich generacji GPU - w modelu renderowania OpenGL dostęp do dowiązanych tekstur odbywał się pośrednio poprzez jednostki teksturujące, które były widoczne przez wszystkie wywołania shaderów i nie pozwalały na zmianę dołączonych tekstur podczas wykonywania polecenia rysowania \cite{GPUGEM4}. 

\subsection {Aktualizacja deskryptorów po dowiązaniu}

Domyślnie deskryptory nie mogą być aktualizowane po nagraniu ich dowiązania w buforze poleceń, przez co aplikacja musi mieć kompletną wiedzę o wszystkich używanych zasobach w trakcie nagrywania buforów poleceń.
Nie dotyczy to jednak deskryptorów używających funkcjonalności aktualizacji po dowiązaniu, co pozwala na elastyczniejsze zarządzanie zasobami poprzez odroczenie aktualizacji zbiorów deskryptorów aż do momentu bezpośrednio przed wykonaniem bufora poleceń.

Wsparcie aktualizacji po dowiązaniu dla wybranego typu deskryptora wymaga odpowiedniej funkcjonalności urządzenia Vulkan 1.2:
\begin{itemize}
	\item {\textit{descriptorBindingUniformBufferUpdateAfterBind}}: bufory uniform,
	\item {\textit{descriptorBindingSampledImageUpdateAfterBind}}: próbkowalne obrazów,
	\item {\textit{descriptorBindingStorageBufferUpdateAfterBind}}: bufory magazynowe,
	\item {\textit{descriptorBindingStorageImageUpdateAfterBind}}: obrazy magazynowe.
\end{itemize}
W czasie pisania pracy powyższe funkcjonalności są wspierane przez ponad $90\%$ urządzeń wyłączając bufory uniform niewspierane przez ok. $40\%$ platform \cite{GPUINFO}.

Użycie tej funkcjonalności wprowadza nowe limity \textit{maxPerStageUpdateAfterBindResources} i \textit{maxDescriptorSetUpdateAfterBind*}
zastępujące stare limity \textit{maxPerStageResources} i \textit{maxDescriptorSet*}. Rozszerzenie gwarantuje, że nowe limity są takie same lub znacznie większe od starych limitów. Przykładowo na maszynie testowej limity maxPerStageDescriptorSampledImages i maxDescriptorSetUpdateAfterBindSampledImages to kolejno $65535$ i $1048576$.

Użycie tej funkcjonalności odbywa się poprzez stworzenia puli deskryptorów z flagą \textit{UPDATE\_AFTER\_BIND}. Dowiązania zdefiniowane podczas tworzenia układu zbioru deskryptorów muszą posiadać flagę
\textit{UPDATE\_AFTER\_BIND\_POOL}.

Aplikacja musi zapewnić odpowiednią synchronizację - aktualizowane deskryptory nie mogą być używane przez potok graficzny w momencie aktualizacji.

\subsection {Dowiązanie deskryptora o zmiennej wielkości}

Domyślnie wielkość dowiązania deskryptora jest stałą wartością określoną podczas stworzenia układu zbioru deskryptora.
Ograniczenie to nie dotyczny dowiązań deskryptora o zmiennej wielkości.

Dzięki tej funkcjonalości wielkość zbioru deskryptorów jest niezależna od układu zbioru deskryptorów i jest specyfikowana dopiero podczas tworzenia zbioru deskryptorów, co pozwala to obsługę sytuacji, w której dokładna liczba deskryptorów wymaganych do opisania zasobów nie jest znana podczas tworzenia układu zbioru deskryptoru.

Wsparcie dowiązań o zmiennej wielkości wymaga funkcjonalności urządzenia Vulkan 1.2 \textit{descriptorBindingVariableDescriptorCount}, która w czasie pisania pracy jest wspierana przez ponad $90\%$ urządzeń \cite{GPUINFO}.

Użycie tej funkcjonalności odbywa się poprzez stworzenia układu zbioru deskryptorów, którego ostatnie dowiązanie posiada flagę \textit{VARIABLE\_DESCRIPTOR\_COUNT} i zamiast liczby deskryptorów podawana jest jej górna granica.
Rzeczywista liczba jest ustalana podczas tworzenia zbioru deskryptorów przy użyciu struktury \textit{VkDescriptorSetVariableDescriptorCountAllocateInfo} w łańcuchu \textit{pNext}.

\subsection {Częściowe dowiązania deskryptorów}

Domyślnie wszystkie deskryptory w dołączonym zbiorze deskryptorów nie mogą być w stanie nieprawidłowym i muszą koniecznie być zaktualizowane przez dowiązaniem - jest to nazywane wymogiem statycznego użycia deskryptorów.
Ograniczenie to nie dotyczny częściowo dowiązanych deskryptorów.

Dzięki tej funkcjonalności deskryptory muszą być dynamicznie używane: deskryptory nieużywane przez shadery mogą być nieprawidłowe i dodatkowo mogą być nawet aktualizowane gdy zbior deskryptorów jest używany przez GPU - pamiętając, że dostęp przy użyciu nieprawidłowego deskryptora jest wciąż niezdefiniowanym zachowaniem i aplikacja musi zapewnić odpowiednią synchronizację CPU-GPU.

Wsparcie dowiązań o zmiennej wielkości wymaga funkcjonalności urządzenia Vulkan 1.2 \textit{descriptorBindingPartiallyBound}, która w czasie pisania pracy jest wspierana przez ponad $94\%$ urządzeń \cite{GPUINFO}.

Użycie tej funkcjonalności odbywa się poprzez stworzenia układu zbioru deskryptorów, którego dowiązanie posiada flagę \textit{PARTIALLY\_BOUND\_BIT}.

\subsection {Tablice deskryptorów czasu wykonania}

Domyślnie rozmiar tablicy deskryptorów musi być znany podczas kompilacji shaderów. Przykładowo w języku GLSL jest on podawany w kwalifikatorze układu.
Wymaganie to nie dotyczy tablic deskryptorów czasu wykonania.

Ta funkcjonalność pozwala na deklarację w shaderach tablic deskryptorów których rozmiar nie jest znany podczas kompilacji, co pozwala na kompilację shaderów bez wiedzy o dokładnej liczbie deskryptorów w dowiązaniach.

Wsparcie tablic deskryptorów czasu wykonania wymaga funkcjonalności urządzenia Vulkan 1.2 \textit{runtimeDescriptorArray}, która w czasie pisania pracy jest wspierana przez ponad $94\%$ urządzeń \cite{GPUINFO}.

Użycie tej funkcjonalności odbywa się poprzez pominięcie rozmiaru tablicy w kwalifikatorze układu w kodzie GLSL:
\lstset{language=GLSL}
\begin{lstlisting}
layout(set = y, binding = x) uniform sampler2D texture[];
\end{lstlisting}
Dostęp do zmiennej w GLSL się nie zmienia, ale wygenerowany kod SPIR-V używa rozszerzenia \textit{SPV\_EXT\_descriptor\_indexing} i typu \textit{OpTypeRuntimeArray} zamiast \textit{OpTypeArray}:
\lstset{language=SPIRV}
\begin{lstlisting}
OpCapability Shader
OpCapability RuntimeDescriptorArray
OpExtension "SPV_EXT_descriptor_indexing"
...
OpName %textures2D "textures2D"
...
OpDecorate %textures2D DescriptorSet 0
OpDecorate %textures2D Binding 2
...
%150 = OpTypeImage %float 2D 0 0 0 1 Unknown
%151 = OpTypeSampledImage %150
%_runtimearr_151 = OpTypeRuntimeArray %151
%_ptr_UniformConstant__runtimearr_151 =OpTypePointer UniformConstant %_runtimearr_151
%textures2D =
	OpVariable %_ptr_UniformConstant__runtimearr_151 UniformConstant
\end{lstlisting}

Indeksowanie poza długością tablicy czasu wykonania jest niezdefiniowanym zachowaniem i może skutkować korupcją.

// TODO

\section{Przebiegi renderowania i potoki}

// TODO

\section{Rozszerzenie VK\_EXT\_dynamic\_rendering}

// TODO

\section{Renderowanie bez dowiązań}

Renderowania bez dowiązań to grupa technik mająca na celu minimalizację liczby poleceń rysowania w celu maksymalizacji czasu GPU, który jest spędzany na rzeczywistym renderowaniu, a nie na zmianach stanu pomiędzy poleceniami [COOKBOOK].

W Vulkan renderowanie bez dowiązań jest realizowane poprzez:
\begin{itemize}
	\item maksymalne zmniejszenie liczby alokowanych deskryptorów,
	\item eliminację kosztownego dowiązywania zasobów między poleceniami rysowania,
	\item łączenie // HIRO multidraw
	\item reifikacja sceny // HIRO
	\item umożliwienia GPU bezpośredniego dostępu do buforów i tekstur poprzez niejednolite dynamiczne indeksowanie deskryptorów.
\end{itemize}

// TODO unified geometry buffer

\subsection {Tekstury bez dowiązań}

Wymóg jednolitości podczas dynamicznego indeksowania deskryptorów wywołuje problemy z renderowaniem scen wypełnionych obiektami używającymi różnych tekstur.
Podczas renderowania obiektów każda zmiana uzywanej tekstury wymaga nagrania nowego polecenia rysowania po zmianie używanych deskryptorów poprzez:
\begin{itemize}
	\item dowiązanie całkowicie nowego zbioru deskryptorów: bardzo kosztowaną operacją, ale wymaga wsparcia tylko statycznego indeksowania,
	\item ponowne dowiązanie deskryptora dynamicznego bufora uniform ze zmienionym dynamicznym offsetem,
	\item zmianę jednolitego indeksu używanego do indeksowania deskryptorów poprzez:
	\begin{itemize}
		\item nagranie nowej stałej push,
		\item zmianę bazowego indeksu instancji w poleceniu rysowania pojedyńczej instancji: wbudowania zmienna shaderów \textit{gl\_InstanceIndex} jest niejednolita tylko dla poleceń rysowania renderujacych wiecej niż jedną instancję.
	\end{itemize}
\end{itemize}

Poniższy diagram przedstawia tradycyjne tekstury z dowiązaniami używające jednolitego indeksu w stałej push:
\begin{figure}[H]
	\centering
	\begin{tikzpicture}[node distance=1cm]
		\tikzstyle{command} = [rectangle, minimum width=5cm, minimum height=1cm,text centered, draw=black]
		\tikzstyle{resource} = [rectangle, rounded corners,text centered, draw=black]
		\tikzstyle{uses} = [thick,->,>=stealth]
		\tikzstyle{binds} = [dotted,->,>=stealth]
		
		\begin{scope}[node distance=1mm and 10mm]
			\node (vkCmdBindDescriptorSets) [command] {vkCmdBindDescriptorSets(...)};
			\node (vkCmdBeginRenderingKHR) [command, below = of vkCmdBindDescriptorSets] {vkCmdBeginRenderingKHR(...)};
			\node (vkCmdBindVertexBuffers) [command, below = of vkCmdBeginRenderingKHR] {vkCmdBindVertexBuffers(...)};
			\node (vkCmdBindIndexBuffer) [command, below = of vkCmdBindVertexBuffers] {vkCmdBindIndexBuffer(...)};
			\node (vkCmdBindPipeline) [command, below = of vkCmdBindIndexBuffer] {vkCmdBindPipeline(...)};
			\node (vkCmdPushConstantsX) [command, below = of vkCmdBindPipeline] {vkCmdPushConstants(...)};
			\node (vkCmdDrawIndexedX) [command, below = of vkCmdPushConstantsX] {vkCmdDrawIndexed(...)};
			\node (vkCmdPushConstantsY) [command, below = of vkCmdDrawIndexedX] {vkCmdPushConstants(...)};
			\node (vkCmdDrawIndexedY) [command, below = of vkCmdPushConstantsY] {vkCmdDrawIndexed(...)};
			\node (vkCmdEndRenderingKHR) [command, below = of vkCmdDrawIndexedY] {vkCmdEndRenderingKHR(...)};
		\end{scope}
		\node(commandBuffer)[draw,dotted,fit=(vkCmdBindDescriptorSets) (vkCmdEndRenderingKHR)] {};
		
		\begin{scope}[node distance=1mm and 10mm]
			\node (textureResourceDot1) [right = of vkCmdBindDescriptorSets] {...};
			\node (textureResourceM) [resource, right = 1mm of textureResourceDot1] {VkImage \#m};
			\node (textureResourceDot2) [right = 1mm of textureResourceM] {...};
			\node (textureResourceN) [resource, right = 1mm of textureResourceDot2] {VkImage \#n};
			\node (textureResourceDot3) [right = 1mm of textureResourceN] {...};
		\end{scope}
		\node(textureResources)[draw,dotted,fit=(textureResourceDot1) (textureResourceN) (textureResourceDot3)] {};
		
		\node (fragmentShader) [resource, right = of vkCmdDrawIndexedY] {Shader fragmentów};
		
		\node (textureIdX) [resource, right = of vkCmdPushConstantsX] {Indeks tekstury \#n};
		\node (textureIdY) [resource, right = of textureIdX] {Indeks tekstury \#m};
		
		\draw [binds] (vkCmdBindDescriptorSets) -- (textureResources);
		\draw [binds] (vkCmdBindPipeline) edge[out=0,in=-180] (fragmentShader);
		
		\draw [binds] (vkCmdPushConstantsX) edge[out=0,in=-180] (textureIdX);
		\draw [binds] (vkCmdPushConstantsY) edge[out=0,in=-90] (textureIdY);
		
		\draw [uses] (vkCmdDrawIndexedX) edge[out=0,in=-180] (fragmentShader);
		\draw [uses] (vkCmdDrawIndexedY) edge[out=0,in=-180] (fragmentShader);
		
		\draw [uses] (fragmentShader) edge (textureIdX);
		\draw [uses] (fragmentShader) edge (textureIdY);
		
		\draw [uses] (textureIdX) edge (textureResourceN);
		\draw [uses] (textureIdY) edge (textureResourceM);
		
		\node (commandBufferLabel)[above=0cm of commandBuffer] {\textbf{Bufor poleceń}};
		\node (textureResourcesLabel)[above=0cm of textureResources] {\textbf{Tablica deskryptorów obrazów}};
		
		\path ([xshift=0mm,yshift=0mm]current bounding box.south east)
		node[matrix,anchor=south east,cells={nodes={font=\sffamily,anchor=west}},
		draw,thick,inner sep=1ex]{
			\draw[uses](0,0) -- ++ (0.6,0); & \node{Użycie};\\
			\draw[binds](0,0) -- ++ (0.6,0); & \node{Dowiązanie/Nagranie};\\
		};
		
	\end{tikzpicture}
	\caption{Tradycyjne tekstury z dowiązaniami używające jednolitego indeksu w stałej push}
	\label{bindful_texture_command_buffer}
\end{figure}

Niejednolite dynamiczne indeksowanie pozwala na dowiązanie wszystkich używanych zasobów na początku bufora poleceń i wyemitowanie pojedyńczego polecenia rysowania, którego shadery używją niejednolitego indeksu do pobrania indeksów używanych tekstur z bufora uniform opisującego obiekty na scenie.

Poniższy diagram przedstawia tekstury bez dowiązań używając niejednolitych indeksów instancji:
\begin{figure}[H]
	\centering
	\begin{tikzpicture}[node distance=1cm]
		\tikzstyle{command} = [rectangle, minimum width=5cm, minimum height=1cm,text centered, draw=black]
		\tikzstyle{resource} = [rectangle, rounded corners,text centered, draw=black]
		\tikzstyle{uses} = [thick,->,>=stealth]
		\tikzstyle{generates} = [dashed,->,>=stealth]
		\tikzstyle{binds} = [dotted,->,>=stealth]
		
		\begin{scope}[node distance=1mm and 10mm]
			\node (vkCmdBindDescriptorSets) [command] {vkCmdBindDescriptorSets(...)};
			\node (vkCmdBeginRenderingKHR) [command, below = of vkCmdBindDescriptorSets] {vkCmdBeginRenderingKHR(...)};
			\node (vkCmdBindVertexBuffers) [command, below = of vkCmdBeginRenderingKHR] {vkCmdBindVertexBuffers(...)};
			\node (vkCmdBindIndexBuffer) [command, below = of vkCmdBindVertexBuffers] {vkCmdBindIndexBuffer(...)};
			\node (vkCmdBindPipeline) [command, below = of vkCmdBindIndexBuffer] {vkCmdBindPipeline(...)};
			\node (vkCmdPushConstants) [command, below = of vkCmdBindPipeline] {vkCmdPushConstants(...)};
			\node (vkCmdDrawIndexed) [command, below = of vkCmdPushConstants] {vkCmdDrawIndexed(...)};
			\node (vkCmdEndRenderingKHR) [command, below = of vkCmdDrawIndexed] {vkCmdEndRenderingKHR(...)};
		\end{scope}
		\node(commandBuffer)[draw,dotted,fit=(vkCmdBindDescriptorSets) (vkCmdEndRenderingKHR)] {};
		
		\begin{scope}[node distance=1mm and 10mm]
			\node (textureResourceDot1) [right = of vkCmdBindDescriptorSets] {...};
			\node (textureResourceM) [resource, right = 1mm of textureResourceDot1] {VkImage \#m};
			\node (textureResourceDot2) [right = 1mm of textureResourceM] {...};
			\node (textureResourceN) [resource, right = 1mm of textureResourceDot2] {VkImage \#n};
			\node (textureResourceDot3) [right = 1mm of textureResourceN] {...};
		\end{scope}
		\node(textureResources)[draw,dotted,fit=(textureResourceDot1) (textureResourceN) (textureResourceDot3)] {};
		
		\begin{scope}[node distance=1mm and 10mm]
			\node (instanceDataDot1) [right = of vkCmdBindVertexBuffers] {...};
			\node (instanceDataX) [resource, right = 1mm of instanceDataDot1] {instancja \#x};
			\node (instanceDataDot2) [right = 1mm of instanceDataX] {...};
			\node (instanceDataY) [resource, right = 1mm of instanceDataDot2] {instancja \#y};
			\node (instanceDataDot3) [right = 1mm of instanceDataY] {...};
		\end{scope}
		\node(instanceData)[draw,dotted,fit=(instanceDataDot1) (instanceDataX) (instanceDataDot3)] {};

		
		\node (fragmentShader) [resource, right = of vkCmdDrawIndexed] {Shader fragmentów};
		
		\node (instanceIdX) [resource, above = 15mm of fragmentShader] {Indeks instancji \#x};
		\node (instanceIdY) [resource, right = of instanceIdX] {Indeks instancji \#y};
		\node(instanceIds)[draw,dotted,fit=(instanceIdX) (instanceIdY)] {};
		
		\draw [binds] (vkCmdBindDescriptorSets) -- (textureResources);
		\draw [binds] (vkCmdBindDescriptorSets) edge[out=0,in=-180] (instanceData);
		\draw [binds] (vkCmdBindPipeline) edge[out=0,in=-180] (fragmentShader);
		
		\draw [generates] (vkCmdDrawIndexed) edge[out=0,in=-180] (instanceIds);
		
		\draw [uses] (vkCmdDrawIndexed) edge (fragmentShader);
		
		\draw [uses] (fragmentShader) edge (instanceIdX);
		\draw [uses] (fragmentShader) edge (instanceIdY);
		
		\draw [uses] (instanceIdX) edge (instanceDataX);
		\draw [uses] (instanceIdY) edge (instanceDataY);
		
		\draw [uses] (instanceDataX) edge (textureResourceN);
		\draw [uses] (instanceDataY) edge (textureResourceM);
		
		\node (commandBufferLabel)[above=0cm of commandBuffer] {\textbf{Bufor poleceń}};
		\node (textureResourcesLabel)[above=0cm of textureResources] {\textbf{Tablica deskryptorów obrazów}};
		\node (instanceDataLabel)[below=2mm of instanceData, fill=white] {\textbf{Deskryptor bufora uniform danych instancji}};
		\node (instanceIdsLabel)[below=2mm of instanceIds, fill=white] {\textbf{Indeksy instancji \textit{gl\_InstanceIndex}}};
		
		\path ([xshift=0mm,yshift=0mm]current bounding box.south east)
		node[matrix,anchor=south east,cells={nodes={font=\sffamily,anchor=west}},
		draw,thick,inner sep=1ex]{
			\draw[uses](0,0) -- ++ (0.6,0); & \node{Użycie};\\
			\draw[binds](0,0) -- ++ (0.6,0); & \node{Dowiązanie};\\
			\draw[generates](0,0) -- ++ (0.6,0); & \node{Generacja};\\
		};
	
	\end{tikzpicture}
	\caption{Tekstury bez dowiązań używając niejednolitych indeksów instancji}
	\label{bindless_texture_command_buffer}
\end{figure}

\subsection {Geometria bez dowiązań}

// TODO

\section{Mapowanie tekstur}

// TODO

\section{Oświetlenie}

// TODO

\section{Cieniowanie odroczone}

// TODO

\section{Graf sceny}

// TODO

\section{Graf renderowania}

// TODO