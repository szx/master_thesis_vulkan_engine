\chapter{Podsumowanie}
\label{chap:summary}

Efektem niniejszej pracy jest silnik graficzny pozwalający na renderowanie używając cieniowania odroczonego i innych technik graficznych.
Zrealizowano zamierzone cele dotyczące funkcjonalności silnika oraz zademonstrowano jego działanie poprzez wyrenderowanie przykładowych scen. Przeprowadzono testy wydajnościowe.

\section{Dalszy rozwój}

Projekt w obecnym kształcie nie został wykorzystany w grze komputerowej i powinien być rozpatrywany w kategoriach prototypu.
Rozwój funkcjonalności silnika nie jest zakończony i w czasie realizacji projektu pojawiło się wiele pomysłów na ulepszenia.

Silnik był napisany i przetestowany wyłącznie na systemie Linux, ale wieloplatformowa natura używanych narzędzi i bibliotek powinna pozwolić na dokonanie kompilacji skrośnej.

Renderer odroczony w obecnym kształcie obsługuje tylko nieprzeźroczyste obiekty. Rysowanie obiektów przeźroczystych mogłoby być rozwiązane dodatkowym przebiegiem renderowania wprzód dla posortowanych obiektów z mieszaniem alfa. Bardziej dokładna, lecz wolniejsza, przeźroczystość dla skomplikowanej geometrii może być osiągnięta używając technik przeźroczystości niezależnej od kolejności (ang. order-independent transparency, OIT).

Wszystkie zasoby są jednokrotnie kopiowane z GPU do CPU przed rozpoczęciem pętli głównej renderowania.
Bardziej efektywnym pamięciowo podejściem byłoby przesyłanie strumieniowe zasobów polegające na ładowaniu ich tylko gdy są rzeczywiście używane do renderowania.

Z przesyłaniem strumieniowym dobrze współpracują techniki usuwania niewidocznych powierzchni.
Najprostszą do implementacji jest odrzucanie na CPU obiektów poza bryłą widzenia pozwalające na niewysyłanie nierenderowanej geometrii do GPU.
Znacznie bardziej skomplikowane, ale zgodne z duchem renderowania bez dowiązań, jest usuwanie niewidocznych powierzchni na GPU. Może być ono zrealizowane w Vulkan używając shaderów obliczeń modyfikujących bufory pośrednie używane przez polecenia rysowania.

Model oświetlenia może być rozszerzony o powierzchnie emitujące światło poprzez dodanie tekstur emisyjnych.
Podobnie tekstury skybox mogą być użyte do mapowania środowiskowego symulującego odbicia światła dla połyskliwych powierzchni.

// HIRO V-bufor zamiast G-bufora bo indeksy

